% Options for packages loaded elsewhere
\PassOptionsToPackage{unicode}{hyperref}
\PassOptionsToPackage{hyphens}{url}
%
\documentclass[
]{article}
\usepackage{lmodern}
\usepackage{amssymb,amsmath}
\usepackage{ifxetex,ifluatex}
\ifnum 0\ifxetex 1\fi\ifluatex 1\fi=0 % if pdftex
  \usepackage[T1]{fontenc}
  \usepackage[utf8]{inputenc}
  \usepackage{textcomp} % provide euro and other symbols
\else % if luatex or xetex
  \usepackage{unicode-math}
  \defaultfontfeatures{Scale=MatchLowercase}
  \defaultfontfeatures[\rmfamily]{Ligatures=TeX,Scale=1}
\fi
% Use upquote if available, for straight quotes in verbatim environments
\IfFileExists{upquote.sty}{\usepackage{upquote}}{}
\IfFileExists{microtype.sty}{% use microtype if available
  \usepackage[]{microtype}
  \UseMicrotypeSet[protrusion]{basicmath} % disable protrusion for tt fonts
}{}
\makeatletter
\@ifundefined{KOMAClassName}{% if non-KOMA class
  \IfFileExists{parskip.sty}{%
    \usepackage{parskip}
  }{% else
    \setlength{\parindent}{0pt}
    \setlength{\parskip}{6pt plus 2pt minus 1pt}}
}{% if KOMA class
  \KOMAoptions{parskip=half}}
\makeatother
\usepackage{xcolor}
\IfFileExists{xurl.sty}{\usepackage{xurl}}{} % add URL line breaks if available
\IfFileExists{bookmark.sty}{\usepackage{bookmark}}{\usepackage{hyperref}}
\hypersetup{
  pdftitle={Interactive Map Walkthrough},
  pdfauthor={Nate Ferraro},
  hidelinks,
  pdfcreator={LaTeX via pandoc}}
\urlstyle{same} % disable monospaced font for URLs
\usepackage[margin=1in]{geometry}
\usepackage{graphicx,grffile}
\makeatletter
\def\maxwidth{\ifdim\Gin@nat@width>\linewidth\linewidth\else\Gin@nat@width\fi}
\def\maxheight{\ifdim\Gin@nat@height>\textheight\textheight\else\Gin@nat@height\fi}
\makeatother
% Scale images if necessary, so that they will not overflow the page
% margins by default, and it is still possible to overwrite the defaults
% using explicit options in \includegraphics[width, height, ...]{}
\setkeys{Gin}{width=\maxwidth,height=\maxheight,keepaspectratio}
% Set default figure placement to htbp
\makeatletter
\def\fps@figure{htbp}
\makeatother
\setlength{\emergencystretch}{3em} % prevent overfull lines
\providecommand{\tightlist}{%
  \setlength{\itemsep}{0pt}\setlength{\parskip}{0pt}}
\setcounter{secnumdepth}{-\maxdimen} % remove section numbering

\title{Interactive Map Walkthrough}
\author{Nate Ferraro}
\date{Summer 2020}

\begin{document}
\maketitle

\begin{center}\rule{0.5\linewidth}{0.5pt}\end{center}

\hypertarget{part-1-data-merge-files}{%
\subsubsection{Part 1: Data Merge Files}\label{part-1-data-merge-files}}

Goal: Make a dataset that can be easily updated with new files and
combined based on column criteria

Create a folder of all datasets you wants to pull from.

\begin{itemize}
\tightlist
\item
  The idea is that you want to pull only the cleaned data from the each
  excel doc.
\item
  \textbf{MAKE SURE THAT YOU ARE PULLING THE RIGHT SHEET.}
\item
  Set correct working directory for where all datasets are held.
\end{itemize}

For example,

\begin{verbatim}
setwd("C:/Examples/Datasets/")
\end{verbatim}

I would recommend holding all datasets in one place.

\emph{Some of the datasets needed to have the columns recoded so
accurately pull the correct varaibles (there was a lack of naming
continunity throughout different iterations)}

\begin{verbatim}
#Example Recode Columns
September_2018 <-read_excel("7.REACH_YEM_Dataset_Joint Market Monitoring Initiative (JMMI)_September2018_recode_cols.xlsx", sheet=2)

#Example
May_2020 <-read_excel("27.REACH_YEM_Dataset_Joint Market Monitoring Initiative (JMMI)_May2020.xlsx", sheet = 3)
\end{verbatim}

\begin{itemize}
\tightlist
\item
  Make sure that you are pulling the right sheet, it should be the
  cleaned data. The cleaned data will be analyzed and compiled into
  datasets.
\end{itemize}

\begin{verbatim}
list_df = setNames(lapply(ls(), function(x) get(x)), ls())
list_df_names <- names(list_df)
\end{verbatim}

\begin{itemize}
\tightlist
\item
  This pulls all names in your environment into a list
  \emph{(environment should be empty before adding all the names of all
  data sets)}, we will use this for an lapply later, and basically run
  analysis and cut down each dataset to what we need.
\end{itemize}

\begin{verbatim}
col_name_initial<-c("fuel_gov_origin","wash_gov_origin",colnames(April_2018%>% dplyr::select(starts_with('calc_price_'),contains("cost_cubic_metere"), contains("exchange_rate_result"),starts_with("governorate_"),starts_with("district_"),-contains('market'))))
data_all_JMMI<-as_tibble(data.frame(test="TEST"))

data_all_JMMI[,col_name_initial] <- NA

colnames_pulled_all<-as_tibble(data.frame(JMMI="TEST"))
\end{verbatim}

\begin{itemize}
\item
  For the first part of this code we want to make a blank dataset that
  we will populate later with the data from the lapply, right now we are
  just working to set the column names correctly and populate it with
  the ``TEST'' and will be a single column for now
\item
  Next we want to just change that to JMMI instead of test, probably an
  easier way to do this but I was lazy and just did it this way
\item
  So we have a list of the column names we want to pull from our loaded
  dataset, and tibble we will pull everything into, and now just need to
  create the function to put it all together
\end{itemize}

\begin{verbatim}
name_object<-function(df){
  name<-deparse(substitute((df)))
  return(name)
}

round_df <- function(df, digits) {
  nums <- vapply(df, is.numeric, FUN.VALUE = logical(1))
  
  df[,nums] <- round(df[,nums], digits = digits)
  
  (df)
}
\end{verbatim}

\begin{itemize}
\tightlist
\item
  These are functions that I threw in.
\item
  The name object worked to dynamically name things in the script before
  the lapply
\item
  Round df is super useful and will use for the sheets that come out in
  the end
\end{itemize}

\begin{verbatim}
col_pull<-function(df, list_of_df){
  #name<- names(list_of_df[df])
 
  call1 <-  sys.call(1)
  call1[[1]] <- as.name("names")
  call1 <- call1[1:2]
  nm <- eval(call1)
  name<-nm[!is.na(match(list_of_df,list(df)))]
  
  colnames(df)<-tolower(colnames(df))
  colnames(df)<-gsub("_all","",colnames(df))
  
  df1<-df%>%
    as_tibble()%>%
    dplyr::select(starts_with('calc_price_'),contains("cost_cubic_meter"), contains("exchange_rate_result"),starts_with("governorate_"),starts_with("district_"),starts_with("fuel_gov_origin"),starts_with("wash_gov_origin"))%>%
    #rename(replace=c(colnames(df)=( gsub("_normalized", "_normalised", colnames(df) ) )  ))%>%
    mutate(as_tibble(),jmmi = name)%>%
    map_if(is.factor,as.character)%>%
    as_tibble()
  
      #colnames(df1)<-gsub(".*/","",colnames(df1))
    
    colname_pull<-as_tibble(data.frame( holder = colnames(df1)))
    names(colname_pull)<-name
    colnames_pulled_all<<-as_tibble(rowr::cbind.fill(colnames_pulled_all, colname_pull, fill=NA ))
  
    data_all_JMMI <<- as_tibble(merge(df1,data_all_JMMI,all=T))
  
  #return(data_all_JMMI)
}
\end{verbatim}

Really important function

\begin{itemize}
\tightlist
\item
  First part is pulling the name from the file we are currently looking
  at so ``name'' at the end is the name of the file we are pulling
  things out of, for example ``May\_2019'', we will use this as a mutate
  later to track which observation go for to each data collection
  period.
\item
  Next we make sure all column names are in lower case and all of the
  "\_all" suffixes are taken from the column names. \emph{(We want the
  column names as uniform as possible so that we can find them easily)}
\item
  We create a ``df1'' which is pulling all the columns we need as well
  as mutating a column for telling us which JMMI this data is from, thus
  the need for pulling the correct name before.
\item
  We also make sure that all factors are characters in this step,
  factors are the worst in this case.
\item
  What really is key here is the \textless\textless- which put merges
  the df1 into the ``data\_all\_JMMI'', the \textless\textless{} act as
  a way to push the data to a dataset outside of the function in the
  environment meaning that it continually saves it, so will be merged
  all the dataset together in a quick way
\end{itemize}

\begin{verbatim}
lapply((list_df), col_pull, list_of_df = list_df)
\end{verbatim}

\begin{itemize}
\tightlist
\item
  The lapply is run on the list of data frames we created in the first
  part of this, so are telling R to go back and run through each of the
  data frames on the list that are stored in the local environment, the
  data frames are then run through the above function, with the list\_df
  being specificed in the lapply as a condition for the col\_pull
  function we just created.
\item
  This lapply can take a while so be patient.
\end{itemize}

\begin{verbatim}
toMatch <- c("calc" ,"exchange","cost")
col_to_numeric<-unique(grep(paste(toMatch, collapse = "|"), colnames(data_all_JMMI),value = T))
data_all_JMMI[col_to_numeric] <- sapply(data_all_JMMI[col_to_numeric], as.numeric)
\end{verbatim}

\begin{itemize}
\tightlist
\item
  Make sure that all of the columns that get pulled are numeric
\end{itemize}

\begin{verbatim}
#substitute out the pcodes to standardize the name (taken from JMMI scripting, with csv (utf-8) sheet)
this_script_path<-(dirname(rstudioapi::getActiveDocumentContext()$path))
setwd(this_script_path)
source("./other scripts/add_pcodes.R")

#debug(add.pcodes)
data_all_JMMI<-add.pcodes(data_all_JMMI)
\end{verbatim}

\begin{itemize}
\tightlist
\item
  Get the names of locations and governorates
\end{itemize}

\begin{verbatim}
#change Pcodes for origin governorates
source("./other scripts/gov_code_switch.R")
data_all_JMMI$fuel_gov_origin<-gov_code_switch(data_all_JMMI$fuel_gov_origin)
data_all_JMMI$wash_gov_origin<-gov_code_switch(data_all_JMMI$wash_gov_origin)

#pull in the full modes script
source("./other scripts/full_modes.R")
\end{verbatim}

\begin{itemize}
\tightlist
\item
  This was an attempt to look at the restocking times, it failed, please
  disregard, will look back into it later.
\end{itemize}

\#\#\#\#\#Start Data Analysis\#\#\#\#\#

\begin{verbatim}
#get rid of all districts that have less than three observation
data_all_JMMI <- data_all_JMMI %>%
                        dplyr::group_by(jmmi)%>%
                        dplyr::group_by(district_id)%>%
                        filter(n()>2)%>%
                        as_tibble()

#make the JMMI column act as a date column and begin to sort by that, will be important for when you want to do that national by the previous month
data_all_JMMI$jmmi<-gsub("_"," ", data_all_JMMI$jmmi)
#this is the actual date that will be sorted by with in the server script
data_all_JMMI$jmmi_date <- as.character(as.Date(as.yearmon(as.character(data_all_JMMI$jmmi))))
date_list<-sort(unique(data_all_JMMI$jmmi_date))
\end{verbatim}

\begin{itemize}
\item
  We want to hold our methodology that all datasets have atleast 3
  districts in each data collection time period.
\item
  The dates need to be cleaned up as we will use it to sort the data for
  future analysis, we use the zoo package for the as.yearmon function
  (really useful but can be tough sometimes to use)
\item
\end{itemize}

\begin{verbatim}
data_all_JMMI$country_id<-"YE"

for(i in seq_along(date_list)){
  if (i ==1){
    df1<-data_all_JMMI%>%
      filter(jmmi_date==date_list[i])
    
      
    district_all<-df1%>%
      aggregate_median("district_id")
    
    district_obs<-df1%>%
      dplyr::select("district_id","jmmi","jmmi_date")%>%
      dplyr::count(district_id, jmmi)
  
    governorate_all<-district_all%>%
      aggregate_median("governorate_id")
  
    governorate_obs<-df1%>%
      dplyr::select("governorate_id","jmmi","jmmi_date")%>%
      dplyr::count(governorate_id, jmmi)
    
    national_all<-governorate_all%>%
      aggregate_median("country_id")
    
    national_obs<-df1%>%
      dplyr::select("country_id","jmmi","jmmi_date")%>%
      dplyr::count(country_id, jmmi)
    
      
  }else{
    df1<-data_all_JMMI%>%
      filter(jmmi_date==date_list[i])
    
    df0<-data_all_JMMI%>%
      filter(jmmi_date==date_list[i-1])
    
    df0_pull<-unique(df0$district_id)
    df_dist<-subset(df1, district_id %in% df0_pull)
    
    district_all_alone<-df_dist%>%
      aggregate_median("district_id")
    
    district_all<-bind_rows(district_all_alone,district_all)
    
    district_obs<-df1%>%
      dplyr::select("district_id","jmmi","jmmi_date")%>%
      dplyr::count(district_id,jmmi)%>%
      bind_rows(district_obs)
    
    governorate_all_alone<-district_all_alone%>%
      aggregate_median("governorate_id")
    
    governorate_all<-bind_rows(governorate_all_alone,governorate_all)
    
    governorate_obs<-df1%>%
      dplyr::select("governorate_id","jmmi","jmmi_date")%>%
      dplyr::count(governorate_id,jmmi)%>%
      bind_rows(governorate_obs)
    
    national_all<-governorate_all_alone%>%
      aggregate_median("country_id")%>%
      bind_rows(national_all)
    
    national_obs<-df1%>%
      dplyr::select("country_id","jmmi","jmmi_date")%>%
      dplyr::count(country_id, jmmi)%>%
      bind_rows(national_obs)
    
    print(date_list[i])
  }
}
\end{verbatim}

\begin{itemize}
\tightlist
\item
  This is a massive for loop that runs the analysis for the medians and
  binds them to the larger datasets.
\item
  The construction of the for loop has two sections is the first part is
  whether the dataset is the first possible option, this is because you
  need something to populate the datasets with. Can't combine the lists
  together without the initial list, so need to use the merge tool with
  two datasets
\item
  Will filter based on the order of the list, which should be set by the
  sort and the yearmon from the zoo package, so should be going in some
  type of order.
\item
  The reason that we are using this order is due to the methodology of
  the JMMI, we only try to assess districts that we assessed in the
  previous round, so it is important to go in order. (This is we have
  the subset in the second part of the for loop
  ``df\_dist\textless-subset(df1, district\_id \%in\% df0\_pull)'')
\end{itemize}

\begin{verbatim}
district_final<-dplyr::full_join(district_all,district_obs, by = c("district_id", "jmmi"))
governorate_final<-dplyr::full_join(governorate_all,governorate_obs, by = c("governorate_id", "jmmi"))
national_final<-dplyr::full_join(national_all,national_obs, by = c("country_id", "jmmi"))
\end{verbatim}

\begin{itemize}
\tightlist
\item
  Here we want to know how many observation we did at each
  district/governorate/national level for each time period, so we are
  doing to use the full join by the level and data collection period to
  make it work.
\end{itemize}

\begin{verbatim}
district_final<-district_final[,c("jmmi_date","governorate_name","governorate_id","district_name","district_id","calc_price_petrol","calc_price_diesel","calc_price_bottled_water","calc_price_treated_water","calc_price_soap","calc_price_laundry","calc_price_sanitary","cost_cubic_meter","exchange_rate_result","n")]
colnames(district_final)<-c("date","government_name","government_ID","district_name","district_ID","petrol","diesel","bottled_water","treated_water","soap","laundry_powder","sanitary_napkins","cost_cubic_meter","exchange_rates","num_obs")

governorate_final<-governorate_final[,c("jmmi_date","governorate_name","governorate_id","calc_price_petrol","calc_price_diesel","calc_price_bottled_water","calc_price_treated_water","calc_price_soap","calc_price_laundry","calc_price_sanitary","cost_cubic_meter","exchange_rate_result","n")]
colnames(governorate_final)<-c("date","government_name","government_ID","petrol","diesel","bottled_water","treated_water","soap","laundry_powder","sanitary_napkins","cost_cubic_meter","exchange_rates","num_obs")

national_final<-national_final[,c("jmmi_date","calc_price_petrol","calc_price_diesel","calc_price_bottled_water","calc_price_treated_water","calc_price_soap","calc_price_laundry","calc_price_sanitary","cost_cubic_meter","exchange_rate_result","n")]
colnames(national_final)<-c("date","petrol","diesel","bottled_water","treated_water","soap","laundry_powder","sanitary_napkins","cost_cubic_meter","exchange_rates","num_obs")
\end{verbatim}

\begin{itemize}
\tightlist
\item
  We want to rename the columns so they interact exactly how we want
  with the codes in the global, server, and ui for the interactive map.
\end{itemize}

\begin{verbatim}
final_list<-list("District"=district_final,"Governorate" = governorate_final, "National" = national_final)

this_script_path<-(dirname(rstudioapi::getActiveDocumentContext()$path))
setwd(this_script_path)

write.xlsx(final_list, file = "./data/updated_interactive.xlsx")
\end{verbatim}

\begin{itemize}
\tightlist
\item
  We are writing a list of the data frames we have just made and
  renamed.
\item
  Set a path to a data folder withing the interactive map files so that
  we can pull the data directly and keep it contained moving forward
  (really its was just easier to do it that way than trying to write to
  googlesheets or something)
\item
  Also it self updates each time so you only have to run it and not do
  anything else
\end{itemize}

\begin{verbatim}
#undo after all clear
write.csv(district_final, file = "./data/district_interactive.csv")
write.csv(governorate_final, file = "./data/governorate_interactive.csv")
write.csv(national_final, file = "./data/national_interactive.csv")
\end{verbatim}

\begin{itemize}
\tightlist
\item
  Just added this at the end
\end{itemize}

\hypertarget{recap}{%
\paragraph{Recap:}\label{recap}}

Overall the data merge file was designed to help make data consolidation
easy and painless. You will want to just link into your dataset each new
time and make sure it is on the right sheet of the cleaned data.
Ideally, this should just go when you add a line you run it and all the
data for the map is updated.

Example of lines to add

\begin{verbatim}
Month_Year <-read_excel("dataset.xlsx", sheet = cleaned_data)
\end{verbatim}

\hypertarget{final-notes}{%
\paragraph{Final notes:}\label{final-notes}}

\begin{enumerate}
\def\labelenumi{\arabic{enumi}.}
\tightlist
\item
  Make sure the columns you want to pull are numerics
\item
  Always check file paths and make sure the script is saved in the right
  spot
\item
  Spot check after running, R can be the worst sometimes, if there is an
  issue it will most likely be in the for\_loop
\item
  If you want to add new data columns begin by updating the
  ``col\_pull'' function, that where the inital pulls start, you will
  also need to probably adjust the ``toMatch'' variable, and the column
  names at the end.
\end{enumerate}

\begin{center}\rule{0.5\linewidth}{0.5pt}\end{center}

\hypertarget{part-2-global-files}{%
\subsubsection{Part 2: Global Files}\label{part-2-global-files}}

This will set the global environment and combine the map shapefiles and
the dataset that we just created in the data merge file. \emph{This is
why we saved the data merge file in the same folder structure, so that
it is easy to grab}

There are 2 initial function at the beginning of the script are very
important

\begin{enumerate}
\def\labelenumi{\arabic{enumi}.}
\tightlist
\item
  The legend decreasing function reverses the lengend to allow for a
  respositioning on the map
\item
  The round df, great quick function to rounding all numerics in a
  dataset
\end{enumerate}

\begin{verbatim}
GSh<-read.csv("data/governorate_interactive.csv")%>%
  as_tibble()%>%
  dplyr::select(-X)

  Admin1data <- mutate(GSh, SMEB = as.numeric((soap*10.5+laundry_powder*20+sanitary_napkins*2+as.numeric(cost_cubic_meter)*3.15))) #The SMEB calculation
  Admin1data$SMEB<-round(Admin1data$SMEB,0)
  
  
GSh2<-read.csv("data/district_interactive.csv")%>%
  as_tibble()%>%
  dplyr::select(-X)

  Admin2data <- mutate(GSh2, SMEB = as.numeric((soap*10.5+laundry_powder*20+sanitary_napkins*2+as.numeric(cost_cubic_meter)*3.15))) #The SMEB caluclation
  Admin2data$SMEB<-round(Admin2data$SMEB,0)

GShnat<-read.csv("data/national_interactive.csv")%>%
  as_tibble()%>%
  dplyr::select(-X)

  AdminNatData<-mutate(GShnat,SMEB = as.numeric((soap*10.5+laundry_powder*20+sanitary_napkins*2+as.numeric(cost_cubic_meter)*3.15))) #The SMEB caluclation)
  AdminNatData$SMEB<-round(AdminNatData$SMEB,0)
  
max_date <- max(as.Date(as.yearmon(AdminNatData$date)))  
\end{verbatim}

So from the this bit of code we are pulling in the datasets that we made
previously in the data merge file and building on them

\begin{itemize}
\tightlist
\item
  The reason I put the SMEB calculation here is incase it changes it
  will be easy to just do in one place. (\emph{so as the SMEB calc
  changes be sure to update or add it here}).
\item
  The other thing is to use the round\_df function to keep all data
  uniform moving forward.
\end{itemize}

\begin{verbatim}
#Wrangle Data into appropriate formats
#Governorates
Admin1table<-as.data.frame(Admin1data)
Admin1table$date2<- as.Date(Admin1table$date, format("%d-%b-%y"), tz="UTC")
Admin1table$date2 <- as.Date(as.yearmon(Admin1table$date))

Admin1data_current <- Admin1table %>% #subset only recent month dates to attach to shapefile
  arrange(desc(date2)) %>%
  filter(date2 == max_date)
currentD <- as.character(format(max(Admin1table$date2),"%B %Y")) #define current date for disply in dashboard
Admin1table[4:14] <- sapply(Admin1table[4:14], as.numeric)

#Districts
Admin2table <- as.data.frame(Admin2data)
Admin2table$date2 <- as.Date(as.yearmon(Admin2table$date))

Admin2data_current <- Admin2table %>% #subset only recent month dates to attach to shapefile
  arrange(desc(date2))%>%
  filter(date2 == max_date)
currentD <- as.character(format(max(Admin2table$date2),"%B %Y")) #define current date for disply in dashboard
Admin2table[7:16] <- sapply(Admin2table[7:16], as.numeric)

#National
AdminNatTable<-as.data.frame(AdminNatData)
AdminNatTable$date2 <- as.Date(as.yearmon(AdminNatTable$date))

AdminNatData_current <- AdminNatTable %>% #subset only recent month dates to attach to shapefile
  arrange(desc(date2)) %>%
  filter(date2 == max_date)
currentD <- as.character(format(max(AdminNatTable$date2),"%B %Y"))
  #define current date for disply in dashboard
\end{verbatim}

This part of the code tries to figure out the dates. The dates are
always tricky in R so we need to be really careful with the, as you can
see I was working to get dates into a unified format and create a way
both sort and have markers for what data collection time period it came
from.

If you add more columns and numbers you will have to adapt the sapply
parts of this script to encompass all parts you want to look at.

\begin{verbatim}
Admin1<- readOGR("./www", "YEM_adm1_Governorates")
Admin2<- readOGR("./www", "YEM_adm2_Districts")

Admin1@data$admin1name<-gsub("Amanat Al Asimah", "Sana'a City", Admin1@data$admin1name)
Admin1@data$admin1refn<-gsub("Amanat Al Asimah", "Sana'a City", Admin1@data$admin1refn)
Admin2@data$admin1name<-gsub("Amanat Al Asimah", "Sana'a City", Admin2@data$admin1name)
Admin2@data<- Admin2@data %>% mutate_if(is.factor, as.character) 
\end{verbatim}

We pull in the actual shapefiles from the (wwww) folder and use readOGR
to create objects we can then adapt for future needs.

We needed to rename Amanat Al Asimah as Sana'a City

\begin{verbatim}
##-------------------------- COMBINE TABULAR & SPATIAL DATA----------------------
#Merge data from Google Sheet with Rayon shp file
Rshp <- merge(x=Admin2,y=Admin2data_current, by.x="admin2pcod", by.y= "district_ID")

DistsNumb<-sum(!is.na(Rshp@data$district_name)) #get number of districts covered...

Rshp<-st_simplify(st_as_sf(Rshp), dTolerance = 0.5)
Rshp <- st_transform(x = Rshp, 
                     crs = "+proj=longlat +datum=WGS84 +no_defs +ellps=WGS84 +towgs84=0,0,0")
Rshp<-as(Rshp,"Spatial")
\end{verbatim}

This is an important bit of code because we are merging the spatial data
with our dataset.

\begin{itemize}
\tightlist
\item
  The Rshp file has to go with Admin2 from before and the
  Admin2data\_current \emph{(data Admin 2 is the district level)} it is
  really important to update and make sure by.x and by.y to be the same
  data and have overlap.
\item
  The DistsNumb will be important and just update how many districts
  were in the new dataset.
\item
  The st\_simplify and st\_transform are used to simplify and get into
  the correct projection of WGS84
\item
  The final piece of the code is just to make sure that Rshp is a
  spatial dataset
\end{itemize}

\begin{verbatim}
Admin1<-st_simplify(st_as_sf(Admin1), dTolerance = 0.5)
Admin1<- st_transform(x = Admin1, 
                      crs = "+proj=longlat +datum=WGS84 +no_defs +ellps=WGS84 +towgs84=0,0,0")
Admin1<-as(Admin1,"Spatial")
\end{verbatim}

This bit of code preforms the same actions on the Admin 1 of governorate
level of data. \emph{is less important because we add the data at the
district level, but this will allow for projection of Governorates
overtop of our map}

\begin{verbatim}
# Get polygons centroids
centroids <- as.data.frame(centroid(Admin1))
colnames(centroids) <- c("lon", "lat")
centroids <- data.frame("ID" = 1:nrow(centroids), centroids)

# Create SpatialPointsDataFrame object
coordinates(centroids) <- c("lon", "lat") 
proj4string(centroids) <- sp::proj4string(Admin1) # assign projection
centroids@data <- sp::over(x = centroids, y = Admin1, returnList = FALSE)
centroids1 <- as.data.frame(centroid(Admin1))
colnames(centroids1) <- c("lon", "lat")
centroids@data<- cbind(centroids@data, centroids1)
\end{verbatim}

The centriods allow for the labels to be attached at the correct long
and lat of the shapefiles and create the spatial points dataframe for
the objects

\begin{verbatim}
#YEMEN LABEL
YEMl<- as.data.frame(cbind(48.5164,15.5527))
colnames(YEMl) <- c("lon", "lat")
YEMl <- data.frame("ID" = 1:nrow(YEMl), YEMl)
coordinates(YEMl) <- c("lon", "lat") 
proj4string(YEMl) <- proj4string(Admin1)
UKRl1<- as.data.frame(cbind(48.5164,15.5527))
YEMl@data<-cbind(YEMl@data, "YEMEN", UKRl1 )
colnames(YEMl@data) <- c("index","name","lon", "lat")
\end{verbatim}

Will create the labels for the Yemen polygon

\begin{verbatim}

vars <- c(
  "WASH SMEB"="SMEB",
  "Parallel Exchange Rates"="exchange_rates",
  "Petrol" = "petrol",
  "Diesel" = "diesel",
  "Bottled Water"="bottled_water",
  "Treated Water"="treated_water",
  "Soap"="soap",
  "Laundry Powder"="laundry_powder",
  "Sanitary Napkins"="sanitary_napkins",
  "Water Trucking"= "cost_cubic_meter"
)
\end{verbatim}

Will list out the variables we need for the choice options, will need to
be updated is additional variables are added.

\hypertarget{final-notes-1}{%
\paragraph{Final notes:}\label{final-notes-1}}

\begin{enumerate}
\def\labelenumi{\arabic{enumi}.}
\tightlist
\item
  The Dates have always given me a problem on this script, if there is
  an issues in the dataset check the dates first.
\item
  The Rshp is the most important bit of code, it links both the
  underlying data from the field to the geospatial objects, be careful
  with it
\item
  You can always run this line by line to make sure the data is at the
  correct stage, this will not be as possible the further we get into
  the map, so it will be good to check here and make sure everything is
  where is it supposed to be before moving forward
\end{enumerate}

\begin{center}\rule{0.5\linewidth}{0.5pt}\end{center}

\hypertarget{part-3-the-server}{%
\subsubsection{Part 3: The Server}\label{part-3-the-server}}

This will set the dataset and slice and dice it as needed by the
interactions on the map.

\emph{This is automated, however if there needs to be changes a lot of
work will need to be done on this side of the code }

\hypertarget{server}{%
\paragraph{Server}\label{server}}

\begin{enumerate}
\def\labelenumi{\arabic{enumi}.}
\tightlist
\item
  The Server takes the Rshp dataset that we made and manipulates it into
  the map
\item
  A really important part of this entire section is the numerical
  position of the column and which columns correspond to which
  variables, there probably is an easier way to do this but this is how
  we got it working (if you change the data you will need to update the
  column numbers)
\item
  The server encompasses all parts of the data being used for each part
  of the datasets for each tab within the map. \emph{So you might be
  manipulating the underlying data for a variety of different sections
  at once}
\end{enumerate}

\begin{verbatim}
server<-function(input, output,session) {
  #_________________________create map consisting of several layers and customization___________________
  output$map1 <- renderLeaflet({  #initiate map
    leaflet(options = leafletOptions(minZoom = 4.5)) %>%
      addProviderTiles(providers$Esri.WorldGrayCanvas) %>% #base map, can be changed
      setView(50.911019,15.889618, zoom = 6.5)
    # setMaxBounds( lng1 = -66.9
    #               , lat1 = 37
    #               , lng2 = -66.1
    #               , lat2 = 37.8 )
  })
\end{verbatim}

\begin{itemize}
\tightlist
\item
  This section begins rendering the map, and provides the baselayer, and
  the initial zoom to the map.
\item
  the set view will show where you start the map initially and from
  there can look at what the zoom is. This lat-long here is the point
  that will be in the center of your map, and zoom controls how close
  you are to that point.
\item
  The lat and longs will get the bounds of the map and edges.
\end{itemize}

\begin{verbatim}
 #"observe" inputs to define variables for map colors, titles, legends and subset district data
  observe({
    VARIA <- input$variable1
    #YlOrRd #YlGnBu #RdPu #OrRd #Greys #Greens #viridis #magma
    if (VARIA == "SMEB") {
      dataM<-Rshp[,c(1,5,7,8,10,29,28,30)] #subset exchange rate col
      #mypal<-colorNumeric( palette="RdYlGn", domain=(dataM@data[,6]), na.color = "#9C9D9F", reverse = T)
      mypal<-colorNumeric( palette=(colorRamp(c("#ADFFA5", "#A7383D", "#420A0D"), interpolate="linear")),domain=dataM@data[,6], na.color = "#D0CFCF", reverse = F)
      pLa<-"WASH SMEB: "
      pLa2<-"WASH SMEB"
      en<-" "
      unitA=" YER"
      title_legend<-"WASH SMEB Cost"
    }
    if (VARIA == "exchange_rates") {
      dataM<-Rshp[,c(1,5,7,8,10,27,28,30)] #subset exchange rate col
      mypal<-colorNumeric( palette="Greens", domain=dataM@data[,6], na.color = "#D0CFCF",reverse = F)
      pLa<-"Parallel Exchange Rate: "
      pLa2<-"Parrallel Exchange Rate"
      en<-" "
      unitA=" YER"
      title_legend<-"YER to 1 USD"
    }
    if (VARIA == "petrol") {
      dataM<-Rshp[,c(1,5,7,8,10,19,28,30)] #subset exchange rate col
      mypal<-colorNumeric( palette="YlOrBr", domain=dataM@data[,6], na.color = "#D0CFCF", reverse = F)
      pLa<-"Petrol Price: "
      pLa2<-"Petrol Price"
      en<-" "
      unitA=" YER"
      title_legend<-"Price (1 L)"
    }
    if (VARIA == "diesel") {
      dataM<-Rshp[,c(1,5,7,8,10,20,28,30)] #subset exchange rate col
      #mypal<-colorNumeric( palette="Reds", domain=dataM@data[,6], na.color = "#9C9D9F", reverse = F)
      mypal<-colorNumeric( palette=(colorRamp(c("#FFD7D9", "#FF535B", "#FB000D", "#830007", "#480004"), interpolate="spline")),domain=dataM@data[,6], na.color = "#D0CFCF", reverse = F)
      pLa<-"Diesel Price: "
      pLa2<-"Diesel Price"
      en<-" "
      unitA=" YER"
      title_legend<-"Price (1 L)"
    }
    if (VARIA == "bottled_water") {
      dataM<-Rshp[,c(1,5,7,8,10,21,28,30)] #subset exchange rate col
      #mypal<-colorNumeric( palette="PuBu", domain=dataM@data[,6], na.color = "#9C9D9F", reverse = F)
      mypal<-colorNumeric( palette=(colorRamp(c("#C7C0FF", "#7A6AFF", "#1501B9", "#0A005D", "#050033"), interpolate="spline")),domain=dataM@data[,6], na.color = "#D0CFCF", reverse = F)
      pLa<-"Bottled Water Price: "
      pLa2<-"Bottled Water Price"
      en<-" "
      unitA=" YER"
      title_legend<-"Price (0.75 L)"
    }
    if (VARIA == "treated_water") {
      dataM<-Rshp[,c(1,5,7,8,10,22,28,30)] #subset exchange rate col
      mypal<-colorNumeric( palette=(colorRamp(c("#C3FFFD", "#6EFBF6", "#009F99", "#00504D"), interpolate="linear")),domain=dataM@data[,6], na.color = "#D0CFCF", reverse = F)
      #mypal<-colorNumeric( palette="PuBuGn", domain=dataM@data[,6], na.color = "#9C9D9F", reverse = F)
      pLa<-"Treated Water Price: "
      pLa2<-"Treated Water Price"
      en<-" "
      unitA=" YER"
      title_legend<-"Price (10 L)"
    }
    if (VARIA == "soap") {
      dataM<-Rshp[,c(1,5,7,8,10,23,28,30)] #subset exchange rate col
      mypal<-colorNumeric( palette="RdPu", domain=dataM@data[,6], na.color = "#D0CFCF", reverse = F)
      pLa<-"Soap Price: "
      pLa2<-"Soap Price"
      en<-" "
      unitA=" YER"
      title_legend<-"Price (100 g)"
    }
    if (VARIA == "laundry_powder") {
      dataM<-Rshp[,c(1,5,7,8,10,24,28,30)] #subset exchange rate col
      mypal<-colorNumeric( palette="Purples", domain=dataM@data[,6], na.color = "#D0CFCF",reverse = F)
      pLa<-"Laundry Powder Price: "
      pLa2<-"Laundry Powder Price"
      en<-" "
      unitA=" YER"
      title_legend<-"Price (100 g)"
    }
    if (VARIA == "sanitary_napkins") {
      dataM<-Rshp[,c(1,5,7,8,10,25,28,30)] #subset exchange rate col
      mypal<-colorNumeric( palette="BuPu", domain=dataM@data[,6], na.color = "#D0CFCF", reverse = F)
      pLa<-"Sanitary Napkin Price: "
      pLa2<-"Sanitary Napkin Price"
      en<-" "
      unitA=" YER"
      title_legend<-"Price (10 Pack)"
    }
    if (VARIA == "cost_cubic_meter") {
      dataM<-Rshp[,c(1,5,7,8,10,26,28,30)] #subset exchange rate col
      #mypal<-colorNumeric( palette="Blues", domain=dataM@data[,6], na.color = "#9C9D9F", reverse = F)
      
      pLa<-"Water Trucking Cost per Cubic Meter: "
      pLa2<-"Water Trucking Cost per Cubic Meter"
      en<-" "
      unitA=" YER"
      title<-"Price (Cubic m)"
      title_legend<-title
    
      
      }
\end{verbatim}

VERY VERY IMPORTANT PART This is the colorization and selection of data
within the map. The VARIA is the potential variables that can be
selected, so with a variety of options it is important to disaggregate
the Rshp data with these if statements.

To look at one example: * first thing is to take out the columns we need
and create our own smaller data grouping

\begin{verbatim}
dataM<-Rshp[,c(1,5,7,8,10,26,28,30)]
\end{verbatim}

IT IS VERY IMPORTNAT TO NOTICE THAT THE NUMBERS OF THE COLUMNS WILL HAVE
TO CHANGE AS IN THE Rshp THE COLUMNS ARE CRUTIAL * 1-admin2pcod -
district pcode * 5-admin1name - governorate name * 7-admin1pcod -
governorate pcode * 8-admin2name - district name * 10-admin2fen -
reference to district shapefile * 26-cost\_cubic\_meter - the variable
selected by the if statement, (\emph{this is the variable that will
change from if statement to if statement}) * 28-num\_obs - the number of
observations * 30-date2 - date of the observations

\begin{verbatim}
 mypal<-colorNumeric( palette=(colorRamp(c("#C9C3F8", "#5D52AD", "#FAD962", "#AA9239"), interpolate="linear")),domain=dataM@data[,6], na.color = "#D0CFCF", reverse = T)
\end{verbatim}

This section just makes the palette needed for mapping, it changes for
each map and selection. I would recommend to keep changing as needed
because the distince color palettes do allow the maps to diffientiate
between each other.

\begin{verbatim}
 pLa<-"Water Trucking Cost per Cubic Meter: "
      pLa2<-"Water Trucking Cost per Cubic Meter"
      en<-" "
      unitA=" YER"
      title<-"Price (Cubic m)"
      title_legend<-title
\end{verbatim}

This is all labeling for maps and legends

That leaves that the inital part of the server completed. We will use
this small dataset and labels to populate the map and datasets

\begin{verbatim}
   #Have a vector of all of the districts that currently have data for the current month
    dataM_NAs<-dataM@data%>%
      filter(is.na(date2))%>%
      pull(admin2pcod)
    
    #get name of variable selected
    call_name<-colnames(dataM@data[6])
    
    #Need to subset out for with districts have had values in the past that arent in this current month, get that as a list
    Admin2data_out <- Admin2table %>% #subset out recent month dates to attach to shapefile
      filter(district_ID %in% dataM_NAs)%>% #next only keep data from places that have had an observation in the past from the right varialbe
      filter(!is.na(get(call_name)))
    
    Admin2data_out <- Admin2data_out[!duplicated(Admin2data_out$district_name),]
    
    Rshp@data<-Rshp@data%>%
      mutate(alt_dist = admin2pcod %in% Admin2data_out$district_ID)
    Rshp@data$alt_dist<-Rshp@data$alt_dist*1
    Rshp@data$alt_dist<-(dplyr::na_if(Rshp@data$alt_dist,0))
\end{verbatim}

This section is corresponds to the past data that is not in the current
month, this will be important for highlighing the past districts on the
map. This part doenst need to be touched too much.Bascially I made a new
column within the Rshp dataset that we will be able to select from.

\begin{verbatim}
   map1<-leafletProxy("map1") %>%
      clearShapes() %>% #clear polygons, so new ones can be added
      clearControls()%>% #reset zoom etc
\end{verbatim}

This is the beginning of the map and will be set and clear old controls
to make the map fresh each tiem.

\begin{verbatim}
 addLabelOnlyMarkers(centroids, #ADD governorate LABELS
                          lat=centroids$lat, 
                          lng=centroids$lon, 
                          label=as.character(centroids$admin1name),
                          labelOptions = leaflet::labelOptions(
                            noHide = TRUE,
                            interactive = FALSE,
                            direction = "bottom",
                            textOnly = TRUE,
                            offset = c(0, -10),
                            opacity = 0.6,
                            style = list(
                              "color"= "black",
                              "font-size" = "13px",
                              "font-family"= "Helvetica",
                              "font-weight"= 600)
                          )) %>%
\end{verbatim}

This creates the governorate labels, and places them on the centroids
that we built in the global section.

\begin{verbatim}
addLabelOnlyMarkers(YEMl, lat=YEMl$lat, #Add Yemen label 
                          lng=YEMl$lon, 
                          label=as.character(YEMl$name),
                          labelOptions = leaflet::labelOptions(
                            noHide = TRUE,
                            interactive = FALSE,
                            direction = "bottom",
                            textOnly = TRUE,
                            offset = c(0, -10),
                            opacity = 1,
                            style = list(
                              "color"= "black",
                              "font-size" = "24px",
                              "font-family"= "Helvetica",
                              "font-weight"= 800,
                              "letter-spacing"= "3px")
                          )) %>%
\end{verbatim}

This create a nice big ``YEMEN'' so you know that you are looking at
Yemen.

\begin{verbatim}
addPolygons(data= dataM,    # add subsetted district shapefiles
                  color = "grey",
                  weight = 0.8,
                  label= paste(dataM$admin2name," (", pLa, dataM@data[,6],en, ")"),
                  opacity = 1.0,
                  smoothFactor = 0.8,
                  fill = TRUE,
                  fillColor =  (~mypal((dataM@data[,6]))),#custom palette
                  fillOpacity = .8,
                  layerId = ~admin2pcod,
                  highlightOptions = highlightOptions(color = "black", weight = 2,
                                                      bringToFront = FALSE, sendToBack = FALSE),
                  popup = paste0(dataM$admin2name, "<br>",'<h7 style="color:black;">',
                                 pLa, "<b>"," ", dataM@data[,6],unitA, "</b>", '</h7>'),
      ) %>%
\end{verbatim}

This is the beginning of the polygons and will add the districts that
are present in this months data. As we see the label is using the labels
from the if statements above. The fill color is the mypal that we used
before and created a custom color palette before. If you remember the
\href{mailto:dataM@data}{\nolinkurl{dataM@data}}{[},6{]} is the only
thing that changes for each if statements, so it is the actual data for
each category.

\begin{verbatim}
      addPolygons(data= old_dist_alt_sp,    # this is you clipped data file of previous districts (make sure it is below your main district on or it will not be seen)
                  color = "red",
                  weight = 1.5,
                  label = paste0(old_dist_alt_sp$admin2name,":previous ",pLa2," data present"), #added a different label that pops up
                  opacity = .40,
                  smoothFactor = 0.5,
                  fill = TRUE,
                  fillColor = ~pal_alt(old_dist_alt_sp@data[,5]), #custom palette as stated before
                  fillOpacity = .8,
                  layerId = ~admin2pcod,
                  highlightOptions = highlightOptions(color = "black", weight = 2,
                    bringToFront = FALSE, sendToBack = FALSE),
      )%>%
\end{verbatim}

This is the dataset for the old districts that have data in the past but
no data currently, so that people know that they can select these
polygons and still interact with the data informing them. The color
around them will be a slight red that should make it easy to see and
highlight but not take too much attention away from the current data.
The label is a past that should help to identify the district to the
user.

\begin{verbatim}
  addPolylines(data = Admin1, #add governorate lines for reference
                   weight= 3.25,
                   stroke = T,
                   color = "black",
                   fill=FALSE,
                   fillOpacity = 0.1,
                   opacity = 0.1 )
\end{verbatim}

This just adds some nice solid governorate lines that help diffientiate
the governorates.

\begin{verbatim}
    map1 %>% clearControls()
    
   map1 %>% 
     addLegend_decreasing("topleft", pal = mypal, values =  dataM@data[,6], #update legend to reflect changes in selected district/variable shown
                     labFormat=labelFormat(suffix=unitA),
                     title = title_legend,
                     opacity = 5,
                     decreasing = T) 
\end{verbatim}

This will add a legend that dynamically updates. This
``addLegend\_decreacing'' function was found and added in the global
section, it is a local function. Do not mess with the function in the
global part of the code. This only really makes the legend decrease,
which is insane that I couldnt find an easier way to do this, but such
is.

\begin{verbatim}
    #needed to make a custom label because i hate R shiny https://stackoverflow.com/questions/52812238/custom-legend-with-r-leaflet-circles-and-squares-in-same-plot-legends
    colors<-c("white", "#D3D3D3", "#D3D3D3")
    labels<-c("Districts with previous data", "Governorate borders", "District borders")
    sizes<-c("20", "20", "20")
    shapes<-c("square", "line", "line")
    borders<-c("red", "#2B2B2B" , "#646464")
\end{verbatim}

This is the beginning of the building of our custom legend, because R
shiny is really difficult to work with. So we had to build our own
custom function to make it work, thatnks to the stackoverflow link in
this section

\begin{verbatim}
 addLegendCustom <- function(map, colors, labels, sizes, shapes, borders, opacity = 0.5){
      
    make_shapes <- function(colors, sizes, borders, shapes) {
      shapes <- gsub("circle", "50%", shapes)
      shapes <- gsub("square", "0%", shapes)
      paste0(colors, "; width:", sizes, "px; height:", sizes, "px; border:3px solid ", borders, "; border-radius:", shapes)
    }
    make_labels <- function(sizes, labels) {
      paste0("<div style='display: inline-block;height: ", 
        sizes, "px;margin-top: 4px;line-height: ", 
        sizes, "px;'>", labels, "</div>")
    }
    
    legend_colors <- make_shapes(colors, sizes, borders, shapes)
    legend_labels <- make_labels(sizes, labels)
    
    return(addLegend(map1,"topleft", colors = legend_colors, labels = legend_labels, opacity = 0.5))
  }
\end{verbatim}

This is the custom legend that we built to make sure that our legends
make sense. It will rely on the inputs that we created in the previous
section.

\begin{verbatim}
#add new legend 
    map1 %>% addLegendCustom(colors, labels, sizes, shapes, borders)
  
    #add scale bar
    map1 %>% addScaleBar("topleft", options = scaleBarOptions(maxWidth = 100, metric = T, imperial = T, updateWhenIdle = T))
    
  
  })#end of MAP
\end{verbatim}

This is the final part of the map rendering, we add our custom legend,
and scale bar. Now we have a map that can be dynamically updated based
on which variable is selected.

\#\#\#\#Now we want to take it further and make sure that we can update
side charts and tables based on what the user selects

\begin{verbatim}
 clicked_state<- eventReactive(input$map1_shape_click,{ #capture ID of clicked district
    return(input$map1_shape_click$id)
  })
\end{verbatim}

Really important part of the code, this section pull the data from
whichever polygon the user selects. So the clicked state variable is the
variable seleted by the user. This will disaggregate our data later.

\begin{verbatim}
  clicked_state_gov<-eventReactive(input$map1_shape_click,{
    gov_id<- substr(input$map1_shape_click$id,1,nchar(input$map1_shape_click$id)-2)
    return(gov_id)
  })
  
  dist_data<-reactive({
    dist_dat<-Admin2table[Admin2table$district_ID==clicked_state(),] #strangely reactive objects are stored as functions
    dist_dat  
  })
  
  gov_data<-reactive({
    gov_dat<-Admin1table[Admin1table$government_ID==clicked_state_gov(),] #adding the government data to the dataset
    gov_dat
  })
  
  nat_data<-reactive({
    nat_dat<-AdminNatTable
    nat_dat
  })
\end{verbatim}

This sets the data and filters it as needed based on the clicked state.
This will become the datasets we use for charts and graphs after, and
thus why the graphs and charts will self update when new districts are
selected.

Each of these datasets will play a different role, remember you always
need what is returned at the end of a reactive to make sure it is pulled
correctly.

\begin{verbatim}
  gov_nat_data<-reactive({
    gov_nat_dat<-right_join(gov_data(),nat_data(), by = "date2")
  })
  
  state_data <- reactive({ #subset JMMI data table based on clicked state
    all_dat<-right_join(dist_data(),gov_nat_data(), by = "date2")#using a full join so that the data that was for the other month when district wasnt select is still shown
    all_dat$date<-as.yearmon(all_dat$date2)
    all_dat
  })
\end{verbatim}

These two datasets are combinations of the state and national and
district datasets, they will be important when we do counts of
observattions per districts as well as making sure that the dates line
up.

Once again I hate using dates in R.

\begin{verbatim}
  chartData1<-reactive({  #subset JMMI data table based on variable of interest (soap, water etc)
    if (input$variable1 == "SMEB"){
      r<-state_data()[,c(1:5,16,17,31,43,15,30,42)]
      colnames(r)<-c("date","government_name","government_ID","district_name","district_ID","variableSEL","date2","governorate_val","nat_val","dist_obs","gov_obs","nat_obs")
    }
    
    if (input$variable1 == "exchange_rates"){
      r<-state_data()[,c(1:5,14,17,29,41,15,30,42)]
      colnames(r)<-c("date","government_name","government_ID","district_name","district_ID","variableSEL","date2","governorate_val","nat_val","dist_obs","gov_obs","nat_obs")
    }
    
    if (input$variable1 == "petrol"){
      r<-state_data()[,c(1:5,6,17,21,33,15,30,42)]
      colnames(r)<-c("date","government_name","government_ID","district_name","district_ID","variableSEL","date2","governorate_val","nat_val","dist_obs","gov_obs","nat_obs")
    }
    
    if (input$variable1 == "diesel"){
      r<-state_data()[,c(1:5,7,17,22,34,15,30,42)]
      colnames(r)<-c("date","government_name","government_ID","district_name","district_ID","variableSEL","date2","governorate_val","nat_val","dist_obs","gov_obs","nat_obs")
    }
    
    if (input$variable1 == "bottled_water"){
      r<-state_data()[,c(1:5,8,17,23,35,15,30,42)]
      colnames(r)<-c("date","government_name","government_ID","district_name","district_ID","variableSEL","date2","governorate_val","nat_val","dist_obs","gov_obs","nat_obs")
    }
    
    if (input$variable1 == "treated_water"){
      r<-state_data()[,c(1:5,9,17,24,36,15,30,42)]
      colnames(r)<-c("date","government_name","government_ID","district_name","district_ID","variableSEL","date2","governorate_val","nat_val","dist_obs","gov_obs","nat_obs")
    }
    
    if (input$variable1 == "soap"){
      r<-state_data()[,c(1:5,10,17,25,37,15,30,42)]
      colnames(r)<-c("date","government_name","government_ID","district_name","district_ID","variableSEL","date2","governorate_val","nat_val","dist_obs","gov_obs","nat_obs")
    }
    
    if (input$variable1 == "laundry_powder"){
      r<-state_data()[,c(1:5,11,17,26,38,15,30,42)]
      colnames(r)<-c("date","government_name","government_ID","district_name","district_ID","variableSEL","date2","governorate_val","nat_val","dist_obs","gov_obs","nat_obs")
    }
    
    if (input$variable1 == "sanitary_napkins"){
      r<-state_data()[,c(1:5,12,17,27,39,15,30,42)]
      colnames(r)<-c("date","government_name","government_ID","district_name","district_ID","variableSEL","date2","governorate_val","nat_val","dist_obs","gov_obs","nat_obs")
    }
    
    if (input$variable1 == "cost_cubic_meter"){
      r<-state_data()[,c(1:5,13,17,28,40,15,30,42)]
      colnames(r)<-c("date","government_name","government_ID","district_name","district_ID","variableSEL","date2","governorate_val","nat_val","dist_obs","gov_obs","nat_obs")
    }
    
    r
  })
\end{verbatim}

This bit of code, like the if statements before, will slice up the
dataset based on what is selected.

It is important to recognize that use of column numbers again and the
column names below inidicate which column is which.

The ``variableSEL'' - or variable selected is the variable value for the
selected district.

The output of this part will be stored as Chart1Data

\begin{verbatim}
chartNAME<-reactive({ #define element to be used as title for selected variable
    if (input$variable1 == "SMEB"){
      y="SMEB"
    }
    
    if (input$variable1 == "exchange_rates"){
      y="Parallel Exchange Rate"
    }
    
    if (input$variable1 == "petrol"){
      y= "Petrol Price"
    }
    
    if (input$variable1 == "diesel"){
      y= "Diesel Price"
    }
    
    if (input$variable1 == "bottled_water"){
      y= "Bottled Water Price"
    }
    
    if (input$variable1 == "treated_water"){
      y= "Treated Water Price"
    }
    
    if (input$variable1 == "soap"){
      y="Soap Price"
    }
    
    if (input$variable1 == "laundry_powder"){
      y= "Laundry Powder Price"
    }
    
    if (input$variable1 == "sanitary_napkins"){
      y= "Sanitary Napkins Price"
    }
    
    if (input$variable1 == "cost_cubic_meter"){
      y= "Water Trucking Price"
    }
    
    y
  })
\end{verbatim}

This part is the same concept as the previous if statement section
however, it is only for titles and such.

\begin{verbatim}
#_________________________Create highcharter element which uses dataset filtered by user inputs___________________
  #NEW OUT PUT FOR DATA TO BE SUBBED LATER
  #https://stackoverflow.com/questions/38113507/r-shiny-keep-retain-values-of-reactive-inputs-after-modifying-selection
  #https://stackoverflow.com/questions/57468457/how-can-i-set-the-yaxis-limits-within-highchart-plot
  
  observe({
    updateSelectInput(session = session, inputId = "varDateSelect", choices = chartData1()$date, selected=lapply(reactiveValuesToList(input), unclass)$varDateSelect)
  })
  
  output$hcontainer <- renderHighchart({
    event <- (input$map1_shape_click) #Critical Line!!!
    
    (validate(need(event$id != "",
                  "Please click on a district to display its history.")))

    ChartDat<-chartData1() #define filtered table that is reactive element chartData1
    #y_min<- chartDatMIN()
    #y_max<- chartDatMAX()
    
    chosenD<- paste0(na.omit(unique(ChartDat[,2])),", ",na.omit(unique(ChartDat[,4]))) #TITLE FOR CHART (governorate and district name)
  
    highchart()%>% #high chart
      hc_xAxis(type = "datetime", dateTimeLabelFormats = list(day = '%b %Y')) %>%
        
      #data for national
      hc_add_series(data=ChartDat, type = "line", hcaes(date2, nat_val), color = "dodgerblue", name=paste(chartNAME(),"-National"))%>%
      #data for governorate
      hc_add_series(data=ChartDat, type = "line", hcaes(date2, governorate_val), color = "forestgreen", name=paste(chartNAME(),"-Governorate"))%>%
      #data for district
      hc_add_series(data=ChartDat, type = "line", hcaes(date2, variableSEL), color="#4F4E51", name=paste(chartNAME(),"-District"))%>%
      
      hc_yAxis(tithcle=list(text=paste0(chartNAME()," in YER")), opposite = FALSE
               #,min= as.numeric(y_min), max= as.numeric(y_max)
    )%>%
      hc_title(text=chosenD)%>%
      hc_add_theme(hc_theme_gridlight())%>%
      hc_plotOptions(line = list(
        lineWidth=1.5,
        dataLabels = list(enabled = FALSE)))
    
  })
\end{verbatim}

This is the beginning of the highcharts and construction of dynamic
tables throughout the output. * The first line has to do with the
pulling the data from the input selections * We pulll the vardateselect
as the input id * The date choices are from the chartdata that we
created before * The event is triggered by which ever input shapefile is
selected. Look that hte \#critical line part of the text * The next line
makes a validation checks to make sure that an actual shape file is
selected * We make the title of the graph here and then start organizing
the data by data. * The rest we use highcharter which is basically a
slightly different ggplot2, if you have questions on highcharter look at
?highcharter

\begin{verbatim}
 #BUILDING TABLE  https://stackoverflow.com/questions/32149487/controlling-table-width-in-shiny-datatableoutput
  output$out_table_obs<-DT::renderDataTable({
    
  ChartDat_Date<-chartData1()#make a new dataset to play around with from the original state data one above
  
  #https://stackoverflow.com/questions/40152857/how-to-dynamically-populate-dropdown-box-choices-in-shiny-dashboard
  
  ChartDat_Date_filter<-ChartDat_Date%>% #filter out based on what was selected from the varDateSelect
    filter(date == input$varDateSelect)
  
  chosenD_Date<- paste0(na.omit(unique(ChartDat_Date[,2])),", ",na.omit(unique(ChartDat_Date[,4])))#make a quick title for the data table
  
  mat_date_test<-matrix(c(round(ChartDat_Date_filter[6],2),
                          round(ChartDat_Date_filter[8],2),
                          round(ChartDat_Date_filter[9],2),
                          ChartDat_Date_filter[10],
                          ChartDat_Date_filter[11],
                          ChartDat_Date_filter[12]), 
    nrow=2, ncol=3, byrow = T,
    dimnames=list(c("Median Price (YER)","Number of Markets Assessed"),c(paste0("District: \n",na.omit(unique(ChartDat_Date[,4]))),paste0("Governorate: \n",na.omit(unique(ChartDat_Date[,2]))),"Yemen")))
  
  DT::datatable(mat_date_test,options = list(dom = 't'))
   })
\end{verbatim}

This datatable produces the table at the bottom of the output, it counts
the number of obserations as well as the absolute prices of goods in
that area. * The dataset is from the ChartData1() that we had produced
before. * We then filter the data based on the variable that was
selected * The data is then taken from the second column of the Chart
Data and a matrix is then created. * The matrix is a series of columns
that have been filtered out and the matrix is row = 2 and col = 3 *
Following the production of the matrix we create the columnes and use
paste0 to construct what they are * Finally it is saved in a DT or
datatable to be projected later.

\begin{verbatim}
  #output infobox for the info exchange rate
  output$info_exchange<-renderValueBox({
    exchange_data<-AdminNatTable
    exchange_data$date<-as.yearmon(exchange_data$date)
    exchange_date<-exchange_data%>%
      filter(date == input$varDateSelect)
    exchange_rate<-exchange_date[1,10]
    
    valueBox(
            value = exchange_rate,
            subtitle="YER to 1 USD", 
            icon = icon("dollar"),
            #fill=T, 
            color = "green")
   
  })
\end{verbatim}

This merely creates that a box that will outline what the exchange rate
was for the month that was selected * This is entirely predicated on the
CSS files we uploaded earlier. (mainly the valuebox part)

\begin{verbatim}
  output$text1 <- renderUI({ #customised text elements
    HTML(paste("Coordinate System:  WGS 1984",
               "Administrative boundaries:  OCHA",
               "R Mapping Packages:  leaflet, shiny, highcharter",
               "Please do NOT use Microsoft Edge for best user interaction",
               sep="<br/>"), '<style type="text/css"> .shiny-html-output { font-size: 11px; line-height: 12px;
         font-family: Helvetica} </style>')h
  })
  
   #
  output$text2 <- renderUI({
    HTML(paste("<i>Note: Data displayed on this map should be interpreted as indicative. 
               In addition, designations and boundaries used here do not imply acceptance 
               by REACH partners and associated donors.</i>",
               sep="<br/>"), '<style type="text/css"> .shiny-html-output { font-size: 11px; line-height: 11px;
         font-family: Helvetica} </style>')
  })
  
  output$text3 <- renderText({ #LARGE TEXT ABOVE CHART
    paste(chartNAME(), " ", "Medians Over Time")
  })

  output$text_DT<-renderText({
    paste0(chartNAME(),", Median Monthly Costs, and Number of Markets Assessed")
  })
  
  output$text4 <- renderUI({
    HTML(paste("<i>For more information, please visit our",a("REACH Website", target="_blank", href="https://www.reach-initiative.org"), 
    "or contact us directly at yemen@reach-initiative.org.</i>"), '<style type="text/css"> .shiny-html-output { font-size: 11px; line-height: 11px;
         font-family: Helvetica} </style>')
  })
  
\end{verbatim}

Creates the custom text elements we will put around the map

\begin{verbatim}
#SMEB DATASET  
  output$table_smeb<-DT::renderDataTable({
    #observe({
    #https://stackoverflow.com/questions/50912519/select-the-number-of-rows-to-display-in-a-datatable-based-on-a-slider-input
    time<-input$months
    percent_time<- input$percent/100
    
    
    #time<-6
    national_data_test<-nat_data()
    national_data_test<-AdminNatTable
    national_data_test$date2 <- as.yearmon(national_data_test$date)
    national_data<-arrange(national_data_test,desc(date2))
    
    month_all<-sort(unique(national_data$date2),decreasing = T)
    time_pull<-month_all[time]
    month_list<-month_all[1:match(time_pull,month_all)]
    
    #now have the month_list which we can cut from in the future
    
    national_data_pull<-dplyr::filter(national_data, date2==month_list)%>%
      dplyr::select(-c(date,num_obs,exchange_rates))
    
    
    national_data_pull<-national_data_pull%>%
      reshape2::melt("date2")%>%
      reshape2::dcast(variable ~ date2)%>%
      round_df(.,0)
    
    col_data_pull<-ncol(national_data_pull)
    
    name_perc_change<-paste0(colnames(national_data_pull[col_data_pull]),
                             " percent change from standard SMEB"
                             )
    #Add SMEB base costs
    #https://stackoverflow.com/questions/13502601/add-insert-a-column-between-two-columns-in-a-data-frame
    national_data_pull<-national_data_pull%>%
      add_column(.,  `Standard SMEB Values`= c(365,430,100,120,130,105,525,1825,12000), .after = 1)%>%
      add_column(.,  `Variable`= c("Petrol","Diesel","Bottled water","Treated water","Soap","Laundry powder","Sanitary napkins","Water trucking","SMEB total"), .after = 1)%>%
      dplyr::select(c(-1))%>%
      dplyr::filter(Variable %in% c("Soap","Laundry powder","Sanitary napkins","Water trucking","SMEB total"))
    
    #get number of columns now we will use later in the formatting of the table
    columns_of_data_begin<-ncol(national_data_pull)+1
    
    #get the column number of the percent change for future formatting
    percent_col<-time+2
    col_format_last<-time+1
    
    
    
    #https://duckduckgo.com/?q=dynamic+naming+in+mutate+R&t=brave&ia=web
    national_data_pull<-national_data_pull%>%
      dplyr::mutate_at(., .vars = c(3:ncol(.)),.funs = list(`percent change from standard SMEB`= ~((.-(national_data_pull[,2]))/(national_data_pull[,2]))                                                        ))
    
    #get number of columns now we will use later in the formatting of the table 
    columns_of_data_end<-ncol(national_data_pull)
    
    #get rid of the weird naming from the mutate_at
    names(national_data_pull) <- gsub("_", " ", names(national_data_pull))
    
    #maybe keep for later
    #dplyr::mutate(.,!!name_perc_change := (((.[,col_data_pull]-.[,2])/(.[,2]))))
    
    #Render the output DT
    #https://stackoverflow.com/questions/60659666/changing-color-for-cells-on-dt-table-in-shiny
    #https://blog.rstudio.com/2015/06/24/dt-an-r-interface-to-the-datatables-library/
    DT::datatable(national_data_pull,extensions = c('FixedColumns'), 
                  options = list(searching = F, paging = F, scrollX=T, fixedColumns = list(leftColumns = 2, rightColumns = 0)),
                  rownames = F)%>%
      formatStyle(columns = 2, color = "white", backgroundColor = "grey", fontWeight = "bold")%>%
      DT::formatPercentage(columns = c(columns_of_data_begin:columns_of_data_end),2)%>%
      formatStyle(columns = c(columns_of_data_begin:columns_of_data_end),
                  color = styleInterval(c(-percent_time,percent_time), c('grey', 'black','white')),
                  backgroundColor = styleInterval(c(-percent_time,percent_time), c('#66FF66', 'white','#FA5353')),
                  fontWeight = styleInterval(c(-percent_time,percent_time),c('bold','normal','bold')))
    
    
  })
  
\end{verbatim}

This create the smeb tracking dataset, It observes the inputs selected
by from the UI and creates tables to match * The two main inputs are the
sliders on the third page of the UI, they look at the months you want to
focus on as well as the percent threshold you want. (these are found as
the varaible time and percent\_time) * The code then melts these options
together to create a functional dataset that we can adapt and rename. *
It is important to note the positions of columns as those are the
reasons for many of the format styles or holds on different namings
throughout the table. * This table only encompasses SMEEB items so if
the SMEB changes it will be good to change tthe base (where we add
columns in the inbetween two dataframe.)

\begin{verbatim}
  #Other goods dataset  
  output$table_other<-DT::renderDataTable({
    #observe({
    #https://stackoverflow.com/questions/50912519/select-the-number-of-rows-to-display-in-a-datatable-based-on-a-slider-input
    time<-input$months
    percent_time<- input$percent/100
    
    
    #time<-6
    national_data_test<-nat_data()
    national_data_test<-AdminNatTable
    national_data_test$date2 <- as.yearmon(national_data_test$date)
    national_data<-arrange(national_data_test,desc(date2))
    
    month_all<-sort(unique(national_data$date2),decreasing = T)
    time_pull<-month_all[time]
    month_list<-month_all[1:match(time_pull,month_all)]
    
    #now have the month_list which we can cut from in the future
    
    national_data_pull<-dplyr::filter(national_data, date2==month_list)%>%
      dplyr::select(-c(date,num_obs,exchange_rates))
    
    
    national_data_pull<-national_data_pull%>%
      reshape2::melt("date2")%>%
      reshape2::dcast(variable ~ date2)%>%
      round_df(.,0)
    
    col_data_pull<-ncol(national_data_pull)
    
    name_perc_change<-paste0(colnames(national_data_pull[col_data_pull]),
                             " percent change from standard SMEB"
    )
    #Add SMEB base costs
    #https://stackoverflow.com/questions/13502601/add-insert-a-column-between-two-columns-in-a-data-frame
    national_data_pull<-national_data_pull%>%
      #add_column(.,  `Standard SMEB Values`= c(365,430,100,120,130,105,525,1825,12000), .after = 1)%>%
      add_column(.,  `Variable`= c("Petrol","Diesel","Bottled water","Treated water","Soap","Laundry powder","Sanitary napkins","Water trucking","SMEB total"), .after = 1)%>%
      dplyr::select(c(-1))%>%
      dplyr::filter(Variable %in% c("Petrol","Diesel","Bottled water","Treated water"))
    
    #get number of columns now we will use later in the formatting of the table
    columns_of_data_begin<-ncol(national_data_pull)+1
    
    #get the column number of the percent change for future formatting
    percent_col<-time+2
    col_format_last<-time+1
    
  
   
    #get number of columns now we will use later in the formatting of the table 
    columns_of_data_end<-ncol(national_data_pull)
    
    #get rid of the weird naming from the mutate_at
    names(national_data_pull) <- gsub("_", " ", names(national_data_pull))
    
    #maybe keep for later
    #dplyr::mutate(.,!!name_perc_change := (((.[,col_data_pull]-.[,2])/(.[,2]))))
    
    #Render the output DT
    #https://stackoverflow.com/questions/60659666/changing-color-for-cells-on-dt-table-in-shiny
    #https://blog.rstudio.com/2015/06/24/dt-an-r-interface-to-the-datatables-library/
    DT::datatable(national_data_pull,extensions = c('FixedColumns'), 
                  options = list(searching = F, paging = F, scrollX=T, fixedColumns = list(leftColumns = 1, rightColumns = 0)),
                  rownames = F)
    
  })
\end{verbatim}

This section is the same as the previous but focuses on non-SMEB good,
its has the same triggers but will not highlight. * This has the same
column naming convention and will help to keep everything easy between
the two codes. * There is no real bas here its just an informative table

\hypertarget{final-notes-2}{%
\paragraph{Final notes:}\label{final-notes-2}}

\begin{enumerate}
\def\labelenumi{\arabic{enumi}.}
\tightlist
\item
  The server is always really complicated and can be a mess to work
  with, to make it easier I made this test code to simulate what the
  server is doing at a give time
\end{enumerate}

\begin{verbatim}



  dist_dat<-Admin2table[Admin2table$district_ID=="YE2307",] #strangely reactive objects are stored as functions


  gov_dat<-Admin1table[Admin1table$government_ID=="YE23",] #adding the government data to the dataset


  nat_dat<-AdminNatTable


  gov_nat_dat<-right_join(gov_dat,nat_dat, by = "date2")

  all_dat<-right_join(dist_dat,gov_nat_dat, by = "date2")#using a full join so that the data that was for the other month when district wasnt select is still shown
  all_dat$date<-as.yearmon(all_dat$date2)
  all_dat

  
  r<-all_dat[,c(1:5,16,17,31,43,15,30,42)]
  colnames(r)<-c("date","government_name","government_ID","district_name","district_ID","variableSEL","date2","governorate_val","nat_val","dist_obs","gov_obs","nat_obs")
r  
  
\end{verbatim}

\begin{enumerate}
\def\labelenumi{\arabic{enumi}.}
\setcounter{enumi}{1}
\tightlist
\item
  If you add to the server make sure to just copy and paste and re check
  the column numbers
\item
  Not a bad idea to save a copy of the server when you make changes
\end{enumerate}

\begin{center}\rule{0.5\linewidth}{0.5pt}\end{center}

\hypertarget{parr-4-the-ui}{%
\subsubsection{Parr 4: The UI}\label{parr-4-the-ui}}

The User Interface is kinda complicated, but if you realize it just
broken up into 4 seperate sections it makes it much easier to manage.

\hypertarget{the-four-sections-of-the-ui}{%
\paragraph{The four sections of the
UI}\label{the-four-sections-of-the-ui}}

\begin{enumerate}
\def\labelenumi{\arabic{enumi}.}
\tightlist
\item
  The map layout (main map and what people will see and interact with)
\item
  The methodology section (pretty set dont really need to touch)
\item
  The SMEB Tracker (displays the tables we built at the end of the
  server, and houses the input sliders we referenced previously)
\item
  The Partners (needs to be updated but is basically just alot of copy
  and paste work)
\end{enumerate}

Beginnings of the code:

\begin{verbatim}
# UI UI UI UI UI UI UI UI UI UI UI UI UI UI UI UI UI UI UI UI UI UI UI UI UI UI UI UI UI UI UI UI UI UI UI UI UI UI UI


#DROP DOWN MENU SELECTIONS 
vars1 <- c(
  "WASH SMEB"="SMEB",
  "Parallel Exchange Rates"="exchange_rates",
  "Petrol" = "petrol",
  "Diesel" = "diesel",
  "Bottled Water"="bottled_water",
  "Treated Water"="treated_water",
  "Soap"="soap",
  "Laundry Powder"="laundry_powder",
  "Sanitary Napkins"="sanitary_napkins",
  "Water Trucking"= "cost_cubic_meter"
)

varsDate<- c("Months to Select" = "varDateSelect")

#USER INTERFACE COMPONENTS 
navbarPage(theme= shinytheme("journal"), 
           title=strong(HTML("<span style='font-size:30px'>YEMEN: JOINT MARKET MONITORING INITIATIVE</span>")), # id="nav", #MAIN TITLE
           windowTitle = "REACH: Yemen Joint Market Monitoring Initiative (JMMI)", #Title for browser tab window
           
\end{verbatim}

This is the start of the code we have all of our variables set right
there, as well as what the variable will be called for the date
(varsDate)

The navbarpage is the beginning of the map and all sheets or tabs start
from there. Nara did a great job outlining what each line does in this
code

The map code:

\begin{verbatim}
 ###..................................M A P. . P A G E ..........................................
           tabPanel(strong("JMMI"), #TAB LABEL
                    icon= icon("map-marker"), #TAB ICON
                    div(class="outer",
                        
                        tags$head(
                          # Include our custom CSS
                          includeCSS("AdminLTE.css"),
                          shiny::includeCSS( "bootstrap.css"), #added 
                          includeCSS(path = "shinydashboard.css"),
                          br()#added
                        ),
                        
                        #LEAFLET MAP
                        # If not using custom CSS, set height of leafletOutput to a number instead of percent
                        leafletOutput("map1", width="100%", height="100%"), #BRING IN LEAFLET MAP, object created in server.R
                        tags$head(tags$style(".leaflet-control-zoom { display: none; }
                                              
                                            #controls {height:90vh; overflow-y: auto; }
                                              ")), #remove map zoom controls
                      
                        tags$head(tags$style(
                          type = "text/css",
                          "#controlPanel {background-color: rgba(255,255,255,0.8);}",
                          ".leaflet-top.leaflet-left .leaflet-control {
                           margin-top: 25px;
                         }"
                        )),
                        #https://stackoverflow.com/questions/37861234/adjust-the-height-of-infobox-in-shiny-dashboard
                        
                       
                        #SIDE PANEL
                        absolutePanel(id = "controls", class = "panel panel-default", fixed = TRUE,
                                      draggable = TRUE, top = 60, left = "auto", right = 20, bottom = 1,
                                      width = 500, height = "auto", 
                                      
                                      hr(),
                                      h5("The Yemen Joint Market Monitoring Initiative (JMMI) is a harmonized price monitoriong initiative that focuses on informing
                                      the Water, Sanitation, and Hygiene (WASH) Cluster and the Cash 
                                      and Market Working Group (CMWG) to support humanitarian activies throughout Yemen.  
                                      The JMMI provides an indicative estimation of the prices of WASH and fuel items across districts in Yemen."),
                                      
                                      h5(tags$u("Most recent findings displayed in map are from data collected in ", #DistsNumn and currentD will change based on the most recent JMMI, defined in global.R
                                                tags$strong(DistsNumb), "districts in ", tags$strong(paste0(currentD,"."))),
                                        ("The districts outlined in red indicate that data for the selected item was collected in that district in previous months.")),
                                      
                                      h5("Further details regarding the JMMI methodology and the Survival Minimum Expenditure Basket (SMEB) calculation can be found on the information tab. 
                                         For additional information on supply chains and market-related concerns, please visit the  ",a("REACH Resource Center", target="_blank",    href="https://www.reachresourcecentre.info/country/yemen/cycle/754/#cycle-754"), " to access the monthly situation overviews."),
                                      
                                      hr(),
                                      
                                      #h5(tags$u("Most recent findings displayed in map are from data collected in ", #DistsNumn and currentD will change based on the most recent JMMI, defined in global.R
                                      #   tags$strong(DistsNumb), "districts in ", tags$strong(currentD))),
                                      
                                      selectInput("variable1", h4("Select Variable Below"), vars1, selected = "SMEB"), #linked text
                                      
                                    
                                      h5(textOutput("text3")), #extra small text which had to be customized as an html output in server.r (same with text1 and text 2)
                                      
                                      #HIGH CHART
                                      highchartOutput("hcontainer", height= 300, width = 450),
                                      
                                      #new data table 
                                      hr(),
                                      selectInput(inputId= "varDateSelect", label = h4("Select Month of Data Collection"), choices=NULL, selected = (("varDateSelect"))),#linked date stuff
                                      h5("Please select a district to enable month selection"),
                                      h5(textOutput("text_DT")),
                                      DT::dataTableOutput("out_table_obs",height = "auto", width = "100%"),
                                      
                                      #####Attempt to add an info box
                                      hr(),
                                      h5("Exchange Rate for selected month"),
                                      h5("Please select month to populate the information box"),
                                      fluidRow(valueBoxOutput("info_exchange", width = 12)),
                                      hr(),
                                      #hr(),
                                      
                                      h6(htmlOutput("text1")),
                                      h6(htmlOutput("text2")),
                                      h6(htmlOutput("text4")),
                                      column(width=12, align="center", div(id="cite2", "Funded by: "), img(src='DFID UKAID.png', width= "90px"),img(src='OCHA@3x.png', width= "90px"),
                                             img(src='USAID.png', width= "105px")) #donor logos
                                      
                                      
                        ),
                        
                        
                        tags$div(id="cite",
                                 a(img(src='reach_logoInforming.jpg', height= "40px"), target="_blank", href="http://www.reach-initiative.org"),
                                 img(src='CMWG Logo.jpg', height= "40px", style='padding:1px;border:thin solid black;'),
                                 img(src='washlogo_grey-300DPI.png', height= "40px"))

                        # tags$div(id="cite",
                        #          a(img(src='reach_logoInforming.jpg', width= "200px"), target="_blank", href="http://www.reach-initiative.org"))
                    )
           ),
\end{verbatim}

\begin{itemize}
\tightlist
\item
  The first part about tags\$heads will bring the CSS part of the
  formatting and help make the overall look of the map much nicer and
  more orderly
\item
  The next part is the actual projection of the map and where it zoom
  angles are
\item
  The next area is about the absolute panel within the tab, the absolute
  panel is the panel that projects where things are and the graphs as
  well as the tables that are set for each selected varaible.
\item
  As you can see in this part there is a lot of text and movement of of
  hr() and h5(), the h5() is just the style and size of the text being
  rendered.
\item
  In this table we also put in the highcharter box as well as the the
  table and other outputs we created at the bottome of the server.
  (\emph{this is dynamic and will update as needed (inshallah)})
\item
  At the bottom you can see the logos used for each one and the web
  addresses needed
\item
  The tag cite was made for the bottome ones and used for the jpgs of
  donors, \textbf{all jpgs should be stored in the wwww folder inside
  the project}
\end{itemize}

The methodology code:

\begin{verbatim}
###..................................I N F O. . P A G E ..........................................
           tabPanel(strong("Information"),
                    tags$head(tags$style("{ height:90vh; overflow-y: scroll; }")),

                    icon= icon("info"), #info-circle
                    div(#class="outer",

                        tags$head(
                          # Include our custom CSS
                          includeCSS("styles.css"),
                        style=" { height:90vh; overflow-y: scroll; }
                                              "), 
                        
                        column(width=8,h3("Overview")), #h1- h5 change the header level of the text
                        
                        column(width=7,h5("The  Yemen  Joint  Market  Monitoring  Initiative  (JMMI) is an
                                          initative led by REACH in collaboration with the Water, Sanitation,
                                          and Hygiene (WASH) Cluster  and the Cash and Market Working Group (CMWG)
                                          to support humanitarian cash actors with the harmonization of price
                                          monitoring throughout Yemen. The basket of goods assessed includes eight 
                                          non-food items (NFIs), including fuel, water and hygiene products, 
                                          reflecting the programmatic areas of the WASH Cluster. The JMMI 
                                          tracks all components of the WASH Survival Minimum Expenditure Basket 
                                          (SMEB) since September 2018.")),
                        
                        column(width=8,h3("Methodology")), #h1- h5 change the header level of the text
                        
                        column(width=7,h5("Data was collected through interviews with vendor Key Informants 
                                          (KIs), selected by partner organizations from markets of various sizes 
                                          in both urban and rural areas. To be assessed by the JMMI, markets 
                                          must be either a single permanent market, or a local community where 
                                          multiple commercial areas are located in close proximity to one another. 
                                          When possible, markets/shops are selected within a single geographical 
                                          location, where there is at least one wholesaler operating in the 
                                          market, or multiple areas of commerce within the same geographical 
                                          location when it is too small, to provide a minimum of three price 
                                          quotations per assessed item.", tags$i(tags$strong("Findings are indicative for the assessed 
                                          locations and timeframe in which the data was collected.")))),
                               
                        column(width=8,h3("SMEB Calculation")), #h1- h5 change the header level of the text
                               
                        column(width=7,h5("Each month, enumerators conduct KI interviews with market vendors to collect three price quotations for each item from the same market in each district. 
                                          REACH calculates the WASH SMEB,
                                          which is composed of four median item prices: Soap (1.05 kg), Laundry Powder (2 kg), Sanitary Napkins (20 units) ,and Water Trucking (3.15 m3).",
                                          p(),
                                          p("The calculation of the aggregated median price for districts and governorates is done following a stepped approach. 
                                          Firstly, the median of all the price quotations related to the same market is taken. Secondly, the median quotation from each market is aggregated to calculate the district median. 
                                          Finally, the median quotation from each district is aggregated to calculate the governorate median. "))),
                               
                        
                        column(width=8,h3("About REACH")), #h1- h5 change the header level of the text
                        
                        column(width=7,h5("REACH is a joint initiative that facilitates the development of 
                                          information tools and products that enhance the capacity of aid actors 
                                          to make evidence-based decisions in emergency, recovery and development 
                                          contexts. By doing so, REACH contributes to ensuring that communities 
                                          affected by emergencies receive the support they need. All REACH 
                                          activities are conducted in support to and within the framework of 
                                          inter-agency aid coordination  mechanisms. For more information, please 
                                          visit our",a("REACH Website", target="_blank",    href="https://www.reach-initiative.org"), "or contact us directly 
                                          at yemen@reach-initiative.org.")),
  
                        hr(),
                       p(),
                      p(),
                      hr(),

                        tags$div(id="cite4",
                                 a(img(src='reach_logoInforming.jpg', width= "200px"), target="_blank", href="http://www.reach-initiative.org")))
                    ),
\end{verbatim}

This part is basically just a bunch of text positioned around. So change
text as need if methodology changes.

The SMEB Tracker Table

\begin{verbatim}
  tabPanel(strong("SMEB Tracker"),
                    
                    
                   style=("{overflow-y:auto; }"), 
                   icon= icon("bar-chart"), #info-circle
                   div(tags$head(
                     # Include our custom CSS
                      tags$style(".fa-check {color:#008000}"),
                      tags$style(HTML(".sidebar {height:50vh; overflow-y:auto; }"))
                    ),
                    sidebarLayout(
                      sidebarPanel(
                        sliderInput("months","Number of months displayed", min = 1, max = 24, step = 1,value = 6, ticks = F),
                        h6("Displays the number of months from the most recent dataset"),
                        br(),
                        br(),
                        sliderInput("percent","Percentage change highlighted", min = 1, max = 100, value = 20, tick=F),
                        h6("Is the percent difference desired for the benchmark"),
                        width=2.5),
                      
                      mainPanel(
                        h2("Monthly SMEB Costs and Percentage Change from Standard SMEB Values"),
                       DT::dataTableOutput("table_smeb"),
                       tags$hr(),
                       h2("Monthly Cost for Other non-SMEB goods"),
                        DT::dataTableOutput("table_other")
                       
                       
                      )
                    )
                    
                    
                    
                    
                   )),           
           #conditionalPanel("false", icon("crosshairs")),
#)
\end{verbatim}

This is a normal table the things to look out for here are the
sliderInputs, those are the inputs that go back to the server, so do
make sure that those are always working, otherwise we are only really
messing with a datatable and then republishing it.

The partners

\begin{verbatim}

tabPanel(strong("Partners"),
         
         #style=("{overflow-y:auto; }"), 
         icon= icon("handshake"), #info-circle
         div(tags$head(
               # Include our custom CSS
               tags$style(".fa-check {color:#008000}"),
               tags$style(HTML(".sidebar {height:50vh; overflow-y:auto; }"))
             ),
             
             column(width=8,h3("Partners (Past and Present)")), #h1- h5 change the header level of the text
             column(width=7, h6(tags$i("Check marks indicate that the partner participated in the most recent month’s JMMI"))),
             
             #icon("check", "fa-2x")),
             #list of partners orgs
             column(width=12,align = "center", h5("Agency for Technical Cooperation and Development (ACTED)",icon("check", "fa-2x")),a(img(src='0_acted.png', height= "50px"), target="_blank", href="https://www.acted.org/en/countries/yemen/")),
             column(width=12,align = "center", h5("Adventist Development and Relief Agency (ADRA)", icon("check", "fa-2x")), a(img(src='0_adra.png', height= "50px"), target = "_blank", href="https://adra.org/")),
             column(width=12,align = "center", h5("Al Thadamon Association"), img(src='0_thadamon.jpg', height= "50px")),
             column(width=12,align = "center", h5("Brains for Development (B4D)"),img(src='0_b4d.jpg', height= "50px")),
             column(width=12,align = "center", h5("Creative Youth Forum (CYF)"), a(img(src='0_cyf.jpg', height= "50px"),target="_blank", href="https://www.facebook.com/cyf.org77/")),
             column(width=12,align = "center", h5("Danish Refugee Council (DRC)", icon("check", "fa-2x")), a(img(src='0_drc.png', height= "50px"), target="_blank", href="http://www.drc.dk")),
             column(width=12,align = "center", h5("Generations without Qat (GWQ)", icon("check", "fa-2x")), img(src='0_gwq.png', height= "50px")),
             #column(width=12,align = "center", h5("LLMPO"), img(src='0_cyf.png', height= "50px")),
             column(width=12,align = "center", h5("International Organization for Migration (IOM)", icon("check", "fa-2x")), img(src='0_iom.png', height= "50px")),
             column(width=12,align = "center", h5("Mercy Corps (MC)"), img(src='0_mercy.jfif', height= "50px")),
             column(width=12,align = "center", h5("National Foundation for Development and Humanitarian Response (NFDHR)",icon("check", "fa-2x")), a(img(src='0_nfdhr.png', height= "50px"),target="_blank", href="http://nfdhr.org/")),
             column(width=12,align = "center", h5("National Forum Human Development (NFHD)"), img(src='0_nfhd.png', height= "50px")),
             column(width=12,align = "center", h5("Norweigan Refugee Council (NRC)", icon("check", "fa-2x")), img(src='0_nrc.png', height= "50px")),
             column(width=12,align = "center", h5("Old City Foundation for Development (OCFD)", icon("check", "fa-2x")), img(src='0_ocfd.jpg', height= "50px")),
             column(width=12,align = "center", h5("OXFAM"), img(src='0_oxfam.png', height= "50px")),
             column(width=12,align = "center", h5("Rising Org. for Children Rights Development (ROC)",icon("check", "fa-2x")), a(img(src='0_roc.jpg', height= "50px"), target="_blank", href="https://rocye.org/")),
             column(width=12,align = "center", h5("Sama Al Yemen", icon("check", "fa-2x")), img(src='0_sama.jpg', height= "50px")),
             column(width=12,align = "center", h5("Save the Children (SCI)",icon("check", "fa-2x")), img(src='0_sci.png', height= "50px")),
             column(width=12,align = "center", h5("Sustainable Development Foundation (SDF)"), img(src='0_sdf.jpg', height= "50px")),
             column(width=12,align = "center", h5("Solidarites International (SI)"), img(src='0_si.jpeg', height= "50px")),
             column(width=12,align = "center", h5("Soul Yemen"), img(src='0_soul.jpg', height= "50px")),
             column(width=12,align = "center", h5("Tamdeen Youth Foundation (TYF)",icon("check", "fa-2x")), img(src='0_tyf.png', height= "50px")),
             column(width=12,align = "center", h5("Vision Hope"), img(src='0_vision.png', height= "50px")),
             column(width=12,align = "center", h5("Yemen Family Care Association (YFCA)"), img(src='0_yfca.jpg', height= "50px")),
             column(width=12,align = "center", h5("Yemen Shoreline Development (YSD)"), img(src='0_ysd.jpg', height= "50px")),
             p(),
             hr()

             
))

)
\end{verbatim}

This part should be changed each month to correspond to which partners
are participating in the JMMI

The basic outline of what you do is keep the first part the same

\begin{verbatim}
column(width=12,align = "center", h5("Adventist Development and Relief Agency (ADRA)", icon("check", "fa-2x")), a(img(src='0_adra.png', height= "50px"), target = "_blank", href="https://adra.org/")),
\end{verbatim}

For instance this part stays the same each time

\begin{verbatim}
column(width=12,align = "center",
\end{verbatim}

This next part is changed to the name of the org, with a check mark and
without are list below

\begin{verbatim}
h5("Adventist Development and Relief Agency (ADRA)", icon("check", "fa-2x"))

h5("Adventist Development and Relief Agency (ADRA)")
\end{verbatim}

So this bit of code should be added after the " in the code above to
produce the check (\emph{, icon(``check'', ``fa-2x'')})

The next part is the image (which should be saved in the www folder,
make sure the png or jpg make sense and are listed as such \emph{dont
call a jpg if the picture is a png})

\begin{verbatim}
, a(img(src='0_adra.png', height= "50px"), target = "_blank", href="https://adra.org/")),
\end{verbatim}

First part is the image call, next is the hight its is set at and target
(leave height and target the say they are)

Finally the href is the website that the user will be taken to if they
click onthe image, so update as needed.

\begin{center}\rule{0.5\linewidth}{0.5pt}\end{center}

That is it.

Good luck.

Always save copies.

Best, Nate

\end{document}
